\documentclass[letter]{article}
\usepackage{amsmath}
\usepackage{amsfonts}
\usepackage{amssymb}
\usepackage{ifthen}
\usepackage{fancyhdr}

%%%
% Set up the margins to use a fairly large area of the page
%%%
\oddsidemargin=.2in
\evensidemargin=.2in
\textwidth=6in
\topmargin=0in
\textheight=9.0in
\parskip=.07in
\parindent=0in
\pagestyle{fancy}

%%%
% Set up the header
%%%
\newcommand{\setheader}[6]{
	\lhead{{\sc #1}\\{\sc #2} ({\small \it \today})}
	\rhead{
		{\bf #3} 
		\ifthenelse{\equal{#4}{}}{}{(#4)}\\
		{\bf #5} 
		\ifthenelse{\equal{#6}{}}{}{(#6)}%
	}
}

%%%
% Set up some shortcut commands
%%%
\newcommand{\R}{\mathbb{R}}
\newcommand{\N}{\mathbb{N}}
\newcommand{\Z}{\mathbb{Z}}
\newcommand{\Proj}{\mathrm{proj}}
\newcommand{\Perp}{\mathrm{perp}}
\newcommand{\proj}{\mathrm{proj}}
\newcommand{\Span}{\mathrm{span}}
\newcommand{\Null}{\mathrm{null}}
\newcommand{\Rank}{\mathrm{rank}}
\newcommand{\mat}[1]{\begin{bmatrix}#1\end{bmatrix}}

%%%
% This is where the body of the document goes
%%%
\begin{document}
	\setheader{Math 211 (A01)}{Homework 1}{Jason Siefken}{V00123456}{John Doe}{V00276354}
	\begin{enumerate}
		\item Tell me about vectors.
		\begin{quote}
			Vectors are the greatest things ever!
			We might write them with a little arrow, like $\vec x$,
			we might write them boldface like $\mathbf{x}$, or we might write them with hats $\hat x$.
			We might even decide to write them with components, like
			\[
				\vec e_2 = \mat{0\\1\\0\\0}\in \R^4.
			\]
		\end{quote}

		\item Tell me about matrices.
		\begin{quote}
			Matrices are also great!  They sure do take up a lot of space
			on the page though, so it's useful to give them names.  For example,
			\[
				A=\mat{1&2&3\\4&5&6\\7&8&9}
				\qquad
				\text{and}
				\qquad
				B=\mat{-1&2&3\\4&-5&6\\7&8&-9}.
			\]
			Then we can write matrix equations easily, like 
			$AB\vec x = B^{-1} A^{T}\vec y$.

			We might want to write row operations:
			\[
				 \mat{1&2&3\\4&5&6\\7&8&9}
				 \xrightarrow{R_2\to R_2-4R_1}
				 \mat{1&2&3\\0&-3&-6\\7&8&9}
			\]\[
				 \xrightarrow{R_3\to R_3-7R_1}
				 \mat{1&2&3\\0&-3&-6\\0&-6&-12}
				 \xrightarrow{R_2\leftrightarrow R_3}
				 \mat{1&2&3\\0&-6&-12\\0&-3&-6}.
			\]
		\end{quote}

		\item Show me some \emph{complicated} ``notation.''
		\begin{quote}
			Well, we can do sums like $\sum_{i=1}^{\infty} x_i$ and fractions like
			$\frac{4+z}{\sqrt{3}}$.  We can also make our brackets scale
			to fit the contents inside:
			\[
				\left(
					17\vec x - \sum_{i=-10}^{10^9} \frac{4\pi \vec e_i}{3}
				\right).
			\]
			We can also do set notation, but we have to escape our curly braces:
			\[
				V=\{ x \in \Z: 5\leq x\leq 12\text{ or }x^2\geq 200 \}.
			\]
			There are also special symbols for $\R$, $\Z$, and $\N$
			and we can pick some fancy fonts if we want others like $\mathcal F$
			or $\mathcal G$ or $\mathfrak W$ or $\mathfrak L$.
			For instance, maybe we want to write $\mathcal L:\R^n\to R^m$.
			We can also write fun greek letters like $\alpha$, $\beta$, $\gamma$,
			$\delta$, and their capital letters like $\Gamma$ and $\Delta$.
		\end{quote}
	\end{enumerate}
\end{document}
