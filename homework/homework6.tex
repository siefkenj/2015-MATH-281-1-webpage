\documentclass[letter]{article}
\usepackage{amsmath}
\usepackage{amsfonts}
\usepackage{amssymb}
\usepackage{ifthen}
\usepackage{fancyhdr}
\usepackage{enumitem}

%%%
% Set up the margins to use a fairly large area of the page
%%%
\oddsidemargin=.2in
\evensidemargin=.2in
\textwidth=6in
\topmargin=0in
\textheight=9.0in
\parskip=.07in
\parindent=0in
\pagestyle{fancy}

%%%
% Set up the header
%%%
\newcommand{\setheader}[6]{
	\lhead{{\sc #1}\\{\sc #2} ({\small \it \today})}
	\rhead{
		{\bf #3} 
		\ifthenelse{\equal{#4}{}}{}{(#4)}\\
		{\bf #5} 
		\ifthenelse{\equal{#6}{}}{}{(#6)}%
	}
}

%%%
% Set up some shortcut commands
%%%
\newcommand{\R}{\mathbb{R}}
\newcommand{\N}{\mathbb{N}}
\newcommand{\Z}{\mathbb{Z}}
\newcommand{\Proj}{\mathrm{proj}}
\newcommand{\Perp}{\mathrm{perp}}
\newcommand{\proj}{\mathrm{proj}}
\newcommand{\Span}{\mathrm{span}}
\newcommand{\Null}{\mathrm{null}}
\newcommand{\Rank}{\mathrm{rank}}
\newcommand{\mat}[1]{\begin{bmatrix}#1\end{bmatrix}}
\renewcommand{\d}{\mathrm{d}}

%%%
% This is where the body of the document goes
%%%
\begin{document}
	\setheader{Math 281-1}{Homework 6}{Not for turning in}{}{}{}
	\begin{enumerate}
		\item
		\begin{enumerate}
			\item Find $f$ so that $\vec F(x,y)=\nabla f(x,y)=(x^2,y^2)$ and use this knowledge
				to compute the amount of work done moving along the parabola $y=2x^2$ from $(-1,2)$ 
				to $(2,8)$.
			\item Find $g$ so that $\vec G(x,y)=\nabla g(x,y)=(\frac{y^2}{1+x^2}, 2y\arctan x)$ and use this knowledge
				to compute the amount of work done moving along the parabola $y=2x^2$ from $(-1,2)$ 
				to $(2,8)$.
		\end{enumerate}

		\item Let $\mathcal S$ be the surface of the unit sphere in $\R^3$.  $\mathcal S$ is parameterized
			by
			\[
				\vec p(\theta,\phi) = \Big(f(\theta, \phi), g(\theta, \phi), h(\theta, \phi)\Big)
			\]
			and is oriented outwards.  Additionally you are told that for a vector field $\vec F:\R^3\to\R^3$,
			the flux of $\vec F$ through $\mathcal S$ is given by 
			\[
				\iint \vec F\circ \vec p(\theta,\phi)
				\cdot
				\left(
					\frac{\partial \vec p}{\partial \theta}(\theta,\phi)
					\times
					\frac{\partial \vec p}{\partial \phi}(\theta,\phi)
				\right)
				\,\d\theta \d\phi.
			\]
			\begin{enumerate}
				\item For a fixed $(\theta_0,\phi_0)$, does 
					$\frac{\partial \vec p}{\partial \theta}(\theta_0,\phi_0)
					\times
					\frac{\partial \vec p}{\partial \phi}(\theta_0,\phi_0)$
					point inwards or outwards?  How about 
					$
					\frac{\partial \vec p}{\partial \phi}(\theta_0,\phi_0)
					\times
					\frac{\partial \vec p}{\partial \theta}(\theta_0,\phi_0)
					$?

				\item Using the component functions $f,g$, and $h$, come up with a new
					parameterization of $\mathcal S$ called $\vec q(\theta, \phi)$
					such that
					\[
					\frac{\partial \vec q}{\partial \phi}(\theta_0,\phi_0)
					\times
					\frac{\partial \vec q}{\partial \theta}(\theta_0,\phi_0)
					\]
					points outwards.  (Be creative; draw pictures; think about the
					orientation of tangent vectors).
			\end{enumerate}

		\item \emph{A new way to do volume forms}.  Let's consider polar coordinates
			again.  We know the volume form for polar coordinates is $r\d r\d \theta$,
			which we computed using the geometry of circles.  However, the
			parameterization
			\[
				\vec p(r,\theta)=(r\cos\theta, r\sin\theta,0),
			\]
			parameterizes a surface (the $xy$-plane) in three dimensions in a very similar
			way to polar coordinates.

			Let $R=\{(x,y,z):x^2+y^2\leq 1\text{ and }z=0\}$.
			\begin{enumerate}
				\item Use the parameterization $\vec p$ to set up an integral to find
					the surface area of $R$.  (I know there are easier ways in
					this case, but set up the integral like a surface integral).
				\item Do you see volume form from polar coordinates in your answer
					to part (a)?
				\item Consider the skewed coordinate system from homework 5 given by 
					\[
						x=a-b\qquad\text{and}\qquad y=2a+b.
					\]
					Imagine the $xy$-plane parameterized by $\vec s(a,b)=(a-b,2a+b,0)$.
					What would it look like if you set up a surface integral to compute
					the area of regions of the $xy$-plane?  How does $\left\| \frac{\partial \vec s}{\partial a}
					\times \frac{\partial \vec s}{\partial b}\right\|$ relate to the volume form
					for skewed coordinates?
				\item Consider the stretched polar coordinates from a million assignments ago given
					by
					\[
						x=\rho\cos \theta\qquad\text{and}\qquad y=2\rho\sin\theta.
					\]
					Imagine again that you parameterize the $xy$-plane with stretched
					polar coordinates and decided to do surface integrals.  Where
					and how does the volume form for stretched polar coordinates appear
					in your answer?
				\item Consider weirdo polar coordinates $\mathcal WP$ give by 
					by
					\[
						x=\rho^2\cos \theta^2\qquad\text{and}\qquad y=\rho^2\sin\theta^2.
					\]
					Compute the volume form for weirdo polar coordinates both from the definition
					of the volume form and by pretending its a surface integral.  Which way
					is easier to compute?  Which way makes the most sense in your head?
			\end{enumerate}

		\item \emph{Conservative can be complicated\ldots}Let $\vec f(x,y) = (\frac{-y}{x^2+y^2}, \frac{x}{x^2+y^2})$.
			\begin{enumerate}
				\item Plot the vector field $\vec f$.  Is it conservative?  Why or why not?
				\item Let $F(x,y)=\arctan(y/x)$.  Compute $\nabla F$.  What's going on here?
				\item Let $R_{(x,y)}$ be the square oriented counter clockwise
					with side lengths 1 and lower-left corner
					at the point $(x,y)$.  Find the work done by $\vec f$ on a particle 
					traversing the paths $R_{(1,1)}$, $R_{(-1,1)}$, and $R_{(-4,-3)}$.  
					Will your answer always be the same?  (Doing these integrals by hand
					isn't important.  After breaking them up into appropriate pieces, you may
					use a compute to evaluate, if you like).

				\item Graph the function \[
						Q(x,y) = \text{work done by $\vec f$ traversing }R_{(x,y)}.
					\]
					How many values does this function take?  Is it defined everywhere?
				\item A subset $X\subset \R^2$ is called \emph{simply connected} if any two circles
					in $X$
					can be deformed into each other without having to leave $X$.  If
					$\vec g= \nabla G$ for some $G:\R^2\to\R$ and the domain of $\vec g$ is
					simply connected, then the work done by $\vec g$ traversing
					any closed loop is zero.  Explain how this property relates 
					to the current situation with $\vec f$.
			\end{enumerate}

	\end{enumerate}

\end{document}
