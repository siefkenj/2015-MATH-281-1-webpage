\documentclass[letter]{article}
\usepackage{amsmath}
\usepackage{amsfonts}
\usepackage{amssymb}
\usepackage{ifthen}
\usepackage{fancyhdr}
\usepackage{enumitem}

%%%
% Set up the margins to use a fairly large area of the page
%%%
\oddsidemargin=.2in
\evensidemargin=.2in
\textwidth=6in
\topmargin=0in
\textheight=9.0in
\parskip=.07in
\parindent=0in
\pagestyle{fancy}

%%%
% Set up the header
%%%
\newcommand{\setheader}[6]{
	\lhead{{\sc #1}\\{\sc #2} ({\small \it \today})}
	\rhead{
		{\bf #3} 
		\ifthenelse{\equal{#4}{}}{}{(#4)}\\
		{\bf #5} 
		\ifthenelse{\equal{#6}{}}{}{(#6)}%
	}
}

%%%
% Set up some shortcut commands
%%%
\newcommand{\R}{\mathbb{R}}
\newcommand{\N}{\mathbb{N}}
\newcommand{\Z}{\mathbb{Z}}
\newcommand{\Proj}{\mathrm{proj}}
\newcommand{\Perp}{\mathrm{perp}}
\newcommand{\proj}{\mathrm{proj}}
\newcommand{\Span}{\mathrm{span}}
\newcommand{\Null}{\mathrm{null}}
\newcommand{\Rank}{\mathrm{rank}}
\newcommand{\mat}[1]{\begin{bmatrix}#1\end{bmatrix}}

%%%
% This is where the body of the document goes
%%%
\begin{document}
	\setheader{Math 281-1}{Homework 2}{Due: Thursday, October 8}{}{}{}
	\begin{enumerate}
		\item A rocket follows a straight path.  Its position along the path is $t^2$ 
			meters from the origin at time $t$.  The radius of the exhaust pipe is governed by the
			function $r(t) = 8-t^{1/3}$.
			\begin{enumerate}
				\item The EPA wants an estimate of the total volume of exhaust from $t=0$ to $t=8$
					seconds.  They request you estimate this volume by sampling the radius of the 
					exhaust pipe and the length of the exhaust column at
					$8$ regularly timed intervals.

					Write and evaluate a summation representing the EPA's requested estimate.
				\item Use an integral to produce a more accurate estimate of the total amount of exhaust (you 
					may assume the exhaust in an exhaust column of height $h$ and radius $r$ is 
					$\pi r^2h$).
			\end{enumerate}

		\item You travel, starting from the origin and heading in the positive direction, along the
			$x$-axis with a speed given by $s(x)=\sqrt{x}$ units per second, where $x$ is your position along
			the $x$-axis.  You sample the height of a function as you travel and discover
			$h(t)=(2-t)^2$ where $t$ is time in seconds.
			\begin{enumerate}
				\item How long does it take you to get to $x=10$?
				\item Give a relationship between $x$ and $t$.
				\item Reparameterize $h$ in terms of $x$.
				\item Write an integral formula for the area under $h$ from $x=0$ to $x=10$.
			\end{enumerate}
	\end{enumerate}
	
	Recall that $\text{Work}=\vec F\cdot \vec d$ where $\vec F$ is a force vector and
	$\vec d$ is a displacement vector.
	
	\begin{enumerate}[resume]
		\item A turbulent river pushes a particle at the point $(x,y)$ with a force
			\[
				\vec F(x,y) = (-yx,x).
			\]
			You are pushing a box through the river from $(0,0)$ to $(1,1)$ along a 
			straight path.
			\begin{enumerate}
				\item Does $\vec F(0,0)\cdot (1,1)$ give you the total work done?
					How about $\vec F(1,1)\cdot (1,1)$?  Why or why not?
				\item Suppose you take tiny steps of size $\sqrt{2}/n$ on your way
					from $(0,0)$ to $(1,1)$.  Write a summation to estimate
					the total work done by using $\vec F(x,y)\cdot \vec d$ as
					an approximation of the work done when
					moving from the origin of $\vec d$
					to the tip of $\vec d$.
				\item Write down the integral that results from your summation in
					(b) when you let $n\to \infty$.
				\item What is the total work done?
			\end{enumerate}
		
		\item A vector moves along the path $\vec r(t) = \mat{\sin t\\ 2\cos t\\ t}$ through a force field given by
			$\vec F(x,y,z)=\mat{yz\\ xz\\ xy}$.  Find the total work done between $t=0$ and $t=\pi$.

		\item An \emph{isometric parameterization} of a 2D surface $S$ is a parameterization $p(t,s)$
			that is area and length preserving.  That is, the speed with respect to the first variable
			is $1$, the speed with respect to the second variable is $1$, and the area of the image
			of the square with corners $(\alpha,\beta),(\alpha+1,\beta),(\alpha,\beta+1),(\alpha+1,\beta+1)$ is 1.

			Consider the surface $S\subset \R^3$ parameterized by \[
				s(x,y) = (x^2,y, |y|^{3/2}).
			\]
			Produce an isometric parameterization $p:\R^2\to\R^3$ and verify each 
			property of the parameterization. (Hint, use a computer to visualize the surface
			and imagine it as a piece of paper.  How choose easy direction vectors to 
			parameterize with.  Also, don't forget that $\|\vec a\times \vec b\|$ gives you
			the area of the parallelogram with sides $\vec a$ and $\vec b$.)
	\end{enumerate}

\end{document}
