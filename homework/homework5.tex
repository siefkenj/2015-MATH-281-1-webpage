\documentclass[letter]{article}
\usepackage{amsmath}
\usepackage{amsfonts}
\usepackage{amssymb}
\usepackage{ifthen}
\usepackage{fancyhdr}
\usepackage{enumitem}

%%%
% Set up the margins to use a fairly large area of the page
%%%
\oddsidemargin=.2in
\evensidemargin=.2in
\textwidth=6in
\topmargin=0in
\textheight=9.0in
\parskip=.07in
\parindent=0in
\pagestyle{fancy}

%%%
% Set up the header
%%%
\newcommand{\setheader}[6]{
	\lhead{{\sc #1}\\{\sc #2} ({\small \it \today})}
	\rhead{
		{\bf #3} 
		\ifthenelse{\equal{#4}{}}{}{(#4)}\\
		{\bf #5} 
		\ifthenelse{\equal{#6}{}}{}{(#6)}%
	}
}

%%%
% Set up some shortcut commands
%%%
\newcommand{\R}{\mathbb{R}}
\newcommand{\N}{\mathbb{N}}
\newcommand{\Z}{\mathbb{Z}}
\newcommand{\Proj}{\mathrm{proj}}
\newcommand{\Perp}{\mathrm{perp}}
\newcommand{\proj}{\mathrm{proj}}
\newcommand{\Span}{\mathrm{span}}
\newcommand{\Null}{\mathrm{null}}
\newcommand{\Rank}{\mathrm{rank}}
\newcommand{\mat}[1]{\begin{bmatrix}#1\end{bmatrix}}
\renewcommand{\d}{\mathrm{d}}

%%%
% This is where the body of the document goes
%%%
\begin{document}
	\setheader{Math 281-1}{Homework 5}{Due: Thursday, November 19}{}{}{}
	\begin{enumerate}
		\item Read about spherical coordinates in the Evans text (or another source of your choosing).
		\begin{enumerate}
			\item Write $(\rho, \theta, \phi) = (1,\pi/6,\pi/8)$ in rectangular coordinates.
			\item Write down the volume form for spherical coordinates (you don't need to derive
				it, just write it down).
			\item Let $H=\{(x,y,z): x^2+y^2+z^2 \leq 9\text{ and }z \geq 0\}$.  Describe $H$ geometrically.
			\item Let $f(x,y,z) = 9-x^2-y^2$.  Find $\displaystyle \int_H f\,\d V$.

		\end{enumerate}

		\item A \emph{torus} is the surface of a doughnut.  Let $\mathcal T$ be a torus with 
			major radius $R=3$ and minor radius $r=1$.  To visualize how $R$ and $r$
			relate to the torus, imagine slicing it in half.  Looking from the side,
			you see two circles of radius $r$ and the centers of those two circles
			are separated by the diameter $2R$.
		\begin{enumerate}
			\item Find a parameterization $\vec t:[0,2\pi)\times [0,2\pi)\to \mathcal T\subset \R^3$
				of the torus with the properties that: (i) the speed with respect to
				the first parameter is constant, (ii) the speed with respect to the second
				parameter is constant, and (iii) $\vec t(\theta, 0)$ traces out a circle
				of radius $R-r$.  (Note that although the speed with respect to
				each parameter is constant as that parameter varies and the other parameter
				is held constant, it certainly
				doesn't need to be the same constant for different values of your constant parameter!)
			\item Is your parameterization an isometric parameterization?  Why or why not?
			\item Let $X_1 =\{(\theta, \phi): 0\leq \phi\leq \pi/2\text{ and }0\leq \theta < 2\pi\}$
				and $X_2 =\{(\theta, \phi): \pi/2\leq \phi\leq \pi\text{ and }0\leq \theta < 2\pi\}$.
					You meet a strange baker who looks at your parameterization $\vec t$ and says,
				``I will bake you a doughnut, fine student.  And, I will frost this doughnut!
				But, you have a choice\ldots{}I with either cover the region $\vec t(X_1)$ or
				$\vec t(X_2)$ with a layer of frosting $0.1$in thick.''  Being hungry, you crave
				the most possible frosting.  Which choice should you take, or does it even matter?
			\item Let $Y_1 =\{(\theta, \phi): \phi=\pi/200,2\pi/200,\ldots, 100\pi/200\text{ and }\theta
				=\pi/200,2\pi/200,\ldots, 400\pi/200\}$ and 
				$Y_2 =\{(\theta, \phi): \phi=101\pi/200,102\pi/200,\ldots,200\pi/200\text{ and }\theta
				=\pi/200,2\pi/200,\ldots, 400\pi/200\}$.
				Again the baker announces, ``Fine student, I will bake you a doughnut.  And, I will
				sprinkle this doughnut! But, you get a choice\ldots{}I will either place
				one sprinkle at each point in $\vec t(Y_1)$ or one sprinkle at each point in 
				$\vec t(Y_2)$.''  If you want the most sprinkles, which option should you take, or
				does it even matter?
			\item The baker finally goes to make your doughnut but is sloppy and spills icing
				over the entire top of the doughnut.  However, the icing is slippery and
				deposits on the doughnut with a depth given by $0.1\hat n\cdot \hat z$ where
				$\hat n$ is the unit normal vector to a point on the doughnut (you may assume the doughnut
				is laying on a table and $\hat z$ is the unit vector perpendicular to the
				table).  Does this situation make any physical sense?  Set up an integral to
				find the total amount of icing on the doughnut.  You may use a computer
				to evaluate this integral.
		\end{enumerate}

		\item Surface area is subtle.  So subtle that for a long, long time the prevailing
			mathematical definition of surface area was wrong!  Here's what is was:
			\begin{quote}
				A \emph{triangulation} of a surface $\mathcal S$ is a polyhedron
				$P$ whose faces are all triangles such that the vertices of $P$ lie on
				$\mathcal S$.

				Given a triangle in $\R^3$, its \emph{diameter} is the smallest
				diameter of a disk that contains the triangle.  Given a triangulation
				$P$, the diameter of $P$ is the maximum diameter of every triangular face in $P$.

				The \emph{bogus surface area} of $\mathcal S$ can then be defined
				as follows:  Let $P_n$ be a sequence of triangulations of $\mathcal S$
				whose diameter tends to zero.  The surface area of $\mathcal S$
				is the limit of the surface area of $P_n$ as $n\to\infty$ (if the limit
				exists).
			\end{quote}

			In 1890, H. A. Schwartz showed this definition didn't even work for something as
			simple as a cylinder!

			Let $\mathcal C$ be the cylinder of radius 1 and height 1 and no top and no bottom 
			(i.e., $\mathcal C$ is just the curvy part).

			\begin{enumerate}
				\item Let $A_{m,n}$ be a triangulation of $\mathcal C$ defined as follows:
					Cut $\mathcal C$ into $m$ cylinders of height $1/m$.  Place
					$n$ points $\{p_i\}$ around the top of each cylinder slice
					and $n$ points $\{p_i'\}$ around the bottom each cylinder slice
					where the $i$th point is at the angle $\frac{2\pi i}{n}$.
					Form triangles by connecting the points $p_i, p_i', p_{i+1}$ and
					by connecting the points $p_i',p_{i+1},p_{i+1}'$.

					Convince yourself that every triangle in $A_{m,n}$
					has the same area.
					Find the surface area of the triangulation $A_{m,n}$ as a function
					of $m$ and $n$.  What happens as $m,n\to\infty$?  Does this agree with
					what you know the surface area of a cylinder should be?

				\item  Let $B_{m,n}$ be a triangulation of $\mathcal C$ defined as follows:
					Cut $\mathcal C$ into $m$ cylinders of height $1/m$.  For the $j$th slice,
					place
					$n$ points $\{p_i\}$ around the top of the cylinder 
					and $n$ points $\{p_i'\}$ around the bottom of the cylinder where
					the point $p_i$ is placed at an angle $\frac{2\pi i}{n} + \frac{j\pi}{n}$ and the
					point $p_i'$ is placed at an angle $\frac{2\pi i}{n}+ \frac{(j+1)\pi}{n}$.
					Form triangles by connecting the points $p_i, p_i', p_{i+1}$ and
					by connecting the points $p_i',p_{i+1},p_{i+1}'$.

					Notice that $B_{m,n}$ is like $A_{m,n}$ but twisted a bit.

					Convince yourself that every triangle in $B_{m,n}$ has the same area.
					Find the surface area of the triangulation $B_{m,n}$ as a function
					of $m$ and $n$ (I don't want any approximations here, I want the \emph{exact}
					surface area).  What happens as $m,n\to\infty$?  Remember, this is a 2d limit!
					You may use the approximations
					\[
						\sin \tfrac{1}{k}\approx \tfrac{1}{k}\qquad
						\cos\tfrac{1}{k}\approx 1-\tfrac{1}{2k^2}
					\]
					when $k$ is large.

				\item Is all hope lost?  Can you think of a modification to bogus surface area that
					would make it work?  Propose your own definition
					of surface area.



			\end{enumerate}
		

	\end{enumerate}

\end{document}
